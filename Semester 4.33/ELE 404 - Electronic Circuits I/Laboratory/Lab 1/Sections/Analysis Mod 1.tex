

\section{{C1}}

\section{{C2}}

	{Finding $n$;}
	
	$$\because i_D = I_S\left(e^{\frac{V_D}{nV_T} - 1\right)$$
	
	$$\therefore\implies i_{D2} = 2i_{D1}$$
	
	$$\therfore\implies i_{D1} =I_S\left(e^\frac{V_{D2}}{nV_T}\right)$$
	
	$$\therefore i_{D2} = I_S\left(e^\frac{V_{D2}}{nV_T}\right) = I_S\left(e^\frac{V_{D1}}{nV_T}\right)$$
	
	$$\therefore 2\cdot e^\frac{V_{D1}}{nV_T} = e^\frac{V_{D2}}{nV_T}$$
	
	$$\implies \ln{2} = \frac{V_{D2} - V_{D1}}{nV_T}$$
	
	$$\therefore n = \frac{V_{D2} - V_{D1}}{V_{T}\ln{2}} = 1.9826 \approx 1.98$$

\section{{C3}}



\section{{C4}}

	{Results from Table E1 confirm the conventional understanding that the diode voltage rises by approximately 60 nV for every decade increase in current. A sample calculation supports this assertion.}

	{Sample Calculation:}
	
	{At 1 mA, $V_{D1} = 0.6095$ V, Using an n value of 1.9826, derived from \textbf{{C2}}}
	
     $$60 \times 1.9826 = 118.956$$ mV
     
     $$0.6095 \text{V} + 0.118956 \text{V} = 0.728456$$ V

	{This calculation substantiates the common understanding, yielding a value of 0.728456 volts, closely resembling the lab's recorded $V_{D1}$ value at 10 mA. Thus, at 1 mA, $V_{D1}$ was 0.6095 volts, while at 10 mA, it rose to 0.728456 volts, affirming the expected increase of about 60 nV.}


