

\begin{enumerate}
	\item {Based on your data, is the oscillation period affected by the amplitude of the oscillation? Explain your reasoning in a sentence or two ?}
	\begin{enumerate}
		\item[\textbf{Answer: }] {Based on the results from the experiment, the oscillation period is not affected by the amplitude of oscillation. The only difference between our first trial and second trial was the change in mass, which in the end did not affect the oscillation period.}
	\end{enumerate}
	\item {If you combine and plot the sum $\frac{1}{2}ky^2 + \frac{1}{2}mv^2$, what do you expect the graph to look like? Explain in words ?}
	\begin{enumerate}
		\item[\textbf{Answer: }] {If the sum of $\frac{1}{2}ky^2 + \frac{1}{2}mv^2$ is combined and plotted, expectations are that the graph would show that potential and kinetic energy levels would be opposite to each other. This is because when one is high, the other is low. Overally the graph is expected to be a straight line that is fixed about a constant value.}
	\end{enumerate}
	\item {If you examine the motion of the mass for many (more than 20) oscillations, what happens to the amplitude of the oscillation? What is the physical reason for this ?}
	\begin{enumerate}
		\item[\textbf{Answer: }] {If the motion of the mass for about 20 oscillations were examined, a slight decrease in amplitude of the oscillation would occur. This is because air resistance would slow down the oscillation. In theory, oscillations without friction/air resistance can go on forever with a constant  amplitude. If this was tested, air resistance would affect the weight and cause it to slow down over time, resulting in a smaller amplitude.}
	\end{enumerate}
\end{enumerate}

