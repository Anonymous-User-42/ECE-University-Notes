
\subsection{{Description}}

	{An 8-bit register utilizing D flip-flops serves as a digital storage unit, where each flip-flop corresponds to a single bit of data.} 
	
	{The register is structured with eight D flip-flops arranged in a cascading fashion, forming a sequential chain from the least significant bit (LSB) to the most significant bit (MSB).} 
	
	{The presence of the reset input allows for the complete clearing of the register, initializing all flip-flops to a predetermined state, commonly zero ("00000000" in this circuit design).} 
	
	{Synchronization is achieved through a clock signal, ensuring precise data capture at specified clock edges and mitigating signal propagation delays.} 
	
	{This 8-bit register plays a vital role in our digital logic circuit, as its output is an direct input to the arithmetic logic unit (ALU).}

\subsection{{Truth Table}}

	{An 8-bit register would ideally have 256 possible state values for some arbitary string of inputs "A" \& "B".}
	
	{Our choice of "A" \& "B" are the last 4 digits of my student number (50120\textbf{9136}) in hexadecimal converted to binary.}
	
	{$$\therefore A = (91)_{16} = (10010001)_{2}$$}
	
	{$$\therefore B = (36)_{16} = (00110110)_{2}$$}

	\begin{table}[H]
		\centering
		\begin{tabular}{|c|c|c|c|}
		\hline
		\hline
			\textit{Reset} & \textit{Clock} & \textit{A} & $O(A, Clock, Reset)$ \\ 
		\hline
		\hline
			0 & 0 & 10010001 & 00000000 (If previous input exists, O = 10010001) \\ 
			\hline
			0 & 1 & 10010001 & 10010001 \\ 
			\hline
			1 & 0 & 10010001 & 00000000 \\ 
			\hline
			1 & 1 & 10010001 & 00000000 \\ 
		\hline
		\hline
    		\end{tabular}
    		\caption{Truth Table for the A Register}
	\end{table}
	
	\begin{table}[H]
		\centering
		\begin{tabular}{|c|c|c|c|}
		\hline
		\hline
			\textit{Reset} & \textit{Clock} & \textit{B} & $O(B, Clock, Reset)$ \\ 
		\hline
		\hline
			0 & 0 & 00110110 & 00000000 (If previous input exists, O = 00110110) \\ 
			\hline
			0 & 1 & 00110110 & 00110110 \\ 
			\hline
			1 & 0 & 00110110 & 00000000 \\ 
			\hline
			1 & 1 & 00110110 & 00000000 \\ 
		\hline
		\hline
    		\end{tabular}
    		\caption{Truth Table for the B Register}
	\end{table}

\subsection{{Block Diagram}}

	{}

\subsection{{Timing Diagram}}

	{}
