{This laboratory experiment involves the design and construction of a Graphics Processing Unit (GPU) in a VHDL environment, followed by its implementation on an FPGA board. The GPU is composed of four main components: a Storage Unit with two 8-bit registers (utilizing D-Flip-Flops), an Arithmetic Logic Unit (ALU) core, a Control Unit featuring a Finite State Machine (FSM) and a 4:16 Decoder, and Seven-Segment Display units.}

{The Storage Unit, employing D-Flip-Flops, serves to temporarily store inputs A and B, reading them on a rising clock edge and forwarding the input to the output on the subsequent rising clock edge. The Control Unit, housing the FSM and 4 to 16 Decoder, dictates the microcode delivered to the ALU core, influencing the operation performed on the two 8-bit inputs.}

{The ALU core, guided by the Control Unit, executes one of three sets of nine operations on the input data. The final component, Seven-Segment Displays, visually represents the output from the ALU core, showcasing results derived from operations on inputs A and B. The integration of all these components and their functionalities, along with in-depth discussions on waveforms and circuit diagrams, will be thoroughly explored in the upcoming report.}

{}

{}

