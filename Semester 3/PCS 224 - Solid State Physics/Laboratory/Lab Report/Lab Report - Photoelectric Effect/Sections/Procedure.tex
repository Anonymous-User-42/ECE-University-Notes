

{}

\section{{Procedure I - In Equilibrium}}

{}

\begin{enumerate}
	\item {Connect the spring to the horizontal rod that is attached to the stand. Make sure that it is fastened and secured. Hang the hanger (50g) to the end of the spring.}
	\item {Continue to add the masses to the hanger until the spring begins to extend a bit. Once the masses is hanging freely, measure the distance to the bottom of the of the hager from the top of the table.}
	\item {Repeat these steps for each increasing mass measurement. Make sure to record all data and uncertainties.}
\end{enumerate}

\section{{Procedure II - In Oscillation}}

{}

\begin{enumerate}
	\item {Set-up the Equipment.}
	\begin{enumerate}
	\item {Hang 200g of the masses in the spring. Make sure that everything is secure so it doesn't come loose and fall onto the sensor.}
	\item {Align the sensor directly below the spring. Place a metal cage over the sensor to protect it.}
	\item {Connect the motion detector to the DIG/SONIC1 channel of the interface.}
	\item {Practice a few attempts to get an understanding on how it should work. Lift up the hanger a couple centimeters and release it. This should cause the hanger to oscillate vertically.}
	\item {Click on the button “Collect” to start taking data. It will automatically stop after 5 seconds.}
	\item {A sinusoidal graph should appear. Click the button “Zoom All” if not close enough. If any irregular spikes occur, data may have to be re taken.}
	\end{enumerate}
	\item {As the mass is stabally hanging, zero the sensor by clicking the button on the bottom right side of your screen that shows the current position.}
	\item {Lift the mass about 5 centimeters above its current location and release. Like in practice, it should oscillate vertically.}
	\item {Click the button “Collect” and a graph should appear. The graph represents the mass’s position as a function of time.}
	\item {Click on each maximum and minimum points. Calculate the average amplitude of the graph by adding all these points up and dividing by how many there are. Make sure to include uncertainty.}
	\item {With the position graph that was created, determine the period of the oscillation. Do this by dragging the mouse across two points.}
	\item {Repeat the procedure from steps 2-6 for the same mass that was used, except use a larger amplitude of oscillation. If increasing it is not possible, reduce the amplitude.}
	\item {Repeat the procedure from steps 2-6 except use a total mass of 300g. Make sure to record all data and uncertainties. This graph that will be created will be used for analysis II.}
\end{enumerate}


