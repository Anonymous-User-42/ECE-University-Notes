

{}

%\section{{}}

{}

\begin{enumerate}
	\item {Turn on the mercury vapor light source, and allow it to warm up for about 5 minutes.}
	\item {Turn on the multimeter and plug cables into the multimeter to measure voltage (in
the V and COM) ports. Adjust the dial so that the meter is measuring DC voltage.}
	\item {Check the battery on the photodiode apparatus:}
	\begin{enumerate}
	\item {Plug the black cable into the ground terminal on the photodiode apparatus.}
	\item {Plug the red cable into the “+6V MIN” terminal on the photodiode apparatus. If
the multimeter reads below 6.0V, the battery on the apparatus should be replaced.}
	\item {Unplug the red cable, and plug it into the “-6V MIN” terminal on the photodiode
apparatus. If the multimeter reads above -6.0V (i.e. between -6.0V and 0V), the
battery on the apparatus should be replaced.}
	\end{enumerate}
	\item {Plug in the cables into the “OUTPUT” terminals on the photodiode apparatus using
the same polarity as the multimeter (red to red, black to black)}
	\item {By holding your hand up close to the lens, you should be able to see five colors from
the mercury light spectrum (three blue/purple lines, a green line, and a yellow/orange
line). \\ The pattern is repeated twice on each side of the apparatus. The lens/grating
is manufactured to create light brighter on one side of the apparatus than the other. \\ \\
\textbf{Only take measurements on the brighter side}.}
	\item {Rotate the coupling bar so that the photodiode apparatus is directly across from the
lens/grating. Adjust the position of the lens/grating so that the light is sharply focused.}
	\item {Rotate the coupling bar so that one of the wavelengths falls directly on the opening
into the photodiode apparatus.}
	\item {Rotate the light shield out of the way and
check that light of a single color only is en-
tering the black openings inside the photodi-
ode apparatus. \\ If not, adjust the angle of the
photodiode apparatus using a thumbscrew un-
derneath it. You may also find it necessary to
adjust the position of the lens/grating to focus
the light.}
	\item {Once you are satisfied that a single wavelength of light is well-positioned and focused
on the black openings inside the apparatus, rotate the light shield back into place.}
	\item {If you are measuring a yellow or green line, attach the magnetic yellow or green filter to
the opening on the photodiode apparatus. This helps to filter out other wavelengths.}
\	\item {You may now measure the stopping potential with the multimeter. It may take a few
moments for the reading to stabilize (dimmer lines will take longer to stabilize). \\ Should
you want to take more measurements, you can push the “PUSH TO ZERO” button
on the apparatus which will discharge the apparatus and cause it to start another
measurement. Record your data with an estimate of uncertainty.}
	\item {Repeat steps 7 - 11 for all visible lines on the bright side of the light source. Record
the light color, and the stopping voltage for each.}
\end{enumerate}


