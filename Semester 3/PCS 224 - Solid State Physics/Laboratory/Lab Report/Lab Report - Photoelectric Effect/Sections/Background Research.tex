{Before beginning this investigation we must know the methods and the theory behind the investigation and the due course of action.}

\section{{Hooke's Law}}

	{Hooke's Law refers to the restoration force that acts on an object that is a part of a spring-mass system. It is described mathematically as,}

		$$F_{s} = -kx$$

	{Where, \textit{k} is the spring constant and \textit{x} is the initial displacement of the mass from the spring's equilibrium point. Spring force is always opposite to the direction of the object's displacement.}

	{As change in the displacement can be measured as a instantaneous function of displacement. Mathematically,}

		$$x\left(t\right) = x_{i}\cos{\left(\omega t + \phi\right)}$$

	{Where $x_{i}$ is the initial displacement of the mass on the spring-mass system, $\omega = \sqrt{k/m}$ is the angular frequency of the system and $\phi$ is the phase of the system (zero in our experiment).}

	{Alternatively, we may mathematically represent the instantaneous spring restoring force as,}

		$$F_{s}\left(t\right) = -kx_{i}\cos{\left(\omega t + \phi\right)}$$
		
	{And because our experiment ensures that our system is in phase, $\phi = 0$. Therefore,}		
		
		$$F_{s}\left(t\right) = -kx_{i}\cos{\left(\omega t\right)}$$
		
\section{{Energy of a Spring-Mass System}}

	{Acceleration is defined as a change in velocity, either in magnitude or direction . Because the direction of the velocity changes constantly in uniform circular motion, there is always an acceleration, even if the speed is constant.}
	
	\subsection{{Potential Energy}}
	
		{We know that potential energy is defined mathematically as,}
		
			$$U = W = \int_{x_{i}}^{x_{f}}{F}dx = \int_{x_{i}}^{x_{f}}{-kx}dx = -\left[\frac{1}{2}kx_{f}^2 - \frac{1}{2}kx_{i}^2\right] = U_{i} - U_{f}$$		
		
		{More generally defined as,}		
		
			$$U = \frac{1}{2}kx^2$$		
		
		{Where \textit{k} is the spring constant and \textit{x} is the spring-mass system's displacement from the eqilibrium position.}		
		
	\subsection{{Kinetic Energy}}
	
		{We know for any real general mechanical system, the kinetic energy is mathematically defined as,}
	
			$$K = \frac{1}{2}mv^2$$	
	
		{Where \textit{v} is the instantaneous velocity of the spring-mass system.}	
	
	\subsection{{Total Energy}}
	
		{Total energy is defined as the instantaneous sum of potential and kinetic energies. Mathematically,}
	
			$$E = U + K$$

		{Therfore, using the SHM equations for instantaneous position and velocity we have,}

			$$E = \frac{1}{2}kA^2$$

			
			