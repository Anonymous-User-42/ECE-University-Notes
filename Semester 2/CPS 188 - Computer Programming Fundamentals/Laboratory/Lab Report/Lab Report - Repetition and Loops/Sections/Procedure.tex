

{}

%\section{{Experiment 1}}

{}

\begin{enumerate}
	\item {Each cart's mass, including the weight of any attached items, was measured and recorded, along with an estimate of its uncertainty.}
	\item {The Vernier Computer interface's CH1 and CH2 ports were used to connect the two motion sensors.}
	\item {On both ends of the track, sensors were positioned. The sensors' swivelling heads were turned such that they now faced away from the table.}
	\item {LoggerPro was configured with the necessary options. In order to make sure everything was operating as it should, a few test runs were performed.}
	\item {For the velcro pads on the two carts to adhere to one another when they collided, they were positioned to face one another.}
	\item {We extracted cart "1"'s pre-collision velocity and both carts' post-collision final velocities using LoggerPro.}
	\item {A sufficient amount of mass was added to cart 1 such that its mass increased by at least 50\%. Uncertainty was used to record the new cart's mass to get the initial and final velocities for this set of masses.}
	\item {The mass on cart 1 was taken off and put on cart 2. Uncertainty occurred when the two new masses were recorded to get the initial and final velocities for this set of masses.}
	\item {The magnets on the ends of the carts should be positioned, so they face one another.
When we were confident that our strategy was sound, we hit play on LoggerPro to start collecting data and sent cart "1" hurtling into cart "2".}
	\item {As was done for the perfectly inelastic collision, LoggerPro software was used to measure the velocity of cart "1" before the collision and the final velocity of each cart after the impact.}
	\item {Mass was added to cart 1 to raise its overall mass by 50\%. The cart's updated mass was noted. Repeating the steps, we can see that we measured and recorded this group of masses' initial and final velocities.}
	\item {Take the weight out of cart 1 and put it in cart 2. Cart 2 now has a heavier weight as a result. After recording the new masses of each cart, we repeat the steps to determine the initial and final velocities for this set of masses.}
\end{enumerate}


