\begin{table}[H]
    \centering
        \begin{tabular}{|c|c|}
        \hline
        \hline
        \textit{Physical Quantity} & \textit{Symbol} \\
        \hline
        \hline
        \textit{Spring constant} & \textit{k} \\
        \hline
        \textit{Equilibrium spring length} & \textit{$l_0$} \\
        \hline
        \textit{Extension length} & \textit{x} \\
        \hline
        \textit{Angular displacement/Polar Angle} & \textit{$\theta$} \\
        \hline
        \textit{Radius of spherical mass} & \textit{r} \\
        \hline
        \hline
        \end{tabular}
    \caption{\textit{General physical quantities employed in this investigation}}
\end{table}

\textit{\textbf{Note}: In theory, the equilibrium spring length and the radius of the spherical mass incorporated in research could of any arbitrary value. For the purpose of this investigation, we shall specifically use springs of equilibrium spring length of 1 m and masses of radius $5\times10^{-2}$ m.}

\begin{table}[H]
    \centering
        \begin{tabular}{|c|c|c|}
        \hline
        \hline
        \textit{Independent Variable} & \textit{Dependent Variable} & \textit{Controlled Variable} \\
        \hline
        \hline
        \textit{Time} & \textit{Extension length} & \textit{Temperature} \\
        \hline
        \textit{-} & \textit{Displacement} & \textit{Fluid medium} \\
        \hline
        \textit{-} & \textit{Angular displacement} & \textit{Reference point} \\
        \hline
        \textit{-} & \textit{Absolute frequency} & \textit{-} \\
        \hline
        \hline
        \end{tabular}
    \caption{\textit{Segregation of employed variables as IV, DV or CV}}
\end{table}

\textit{\textbf{Note}: The spring constant, equilibrium length and the radius of the mass is no longer variable as we have defined a set value to it.}

