{Before beginning this investigation we must know the methods and the theory behind the investigation and the due course of action.}

\section{{Sinusoidal Function}}

	{As we are dealing with two sinusoidal waves with equal amplitudes and frequencies moving in opposite directions. These waves can be described mathematically as,}

		$$y_1(x, t) = A\sin(kx-\omega t)$$
		
		$$y_2(x, t) = A\sin(kx+\omega t)$$

	{Where, \textit{k} is the wave number ($k = 2\pi /\lambda$) and $\omega$ is the angular frequency of the system.}

	{Therefore the sum changes of the vertical displacement can be expressed mathematically as,}

		$$\sum{y(x, t)} = y_1(x, t) = A\sin(kx-\omega t) + y_2(x, t) = A\sin(kx+\omega t) = 2A\sin(kx)\cos(\omega t)$$

	{}
		
\section{{Fundamental Frequencies}}

	{There exist's specific frequencies wherein the wave appears to not move and trave in any direction and has a presence of nodes, this interference between two waves traveling in opposite directions is exactly what happens if you send two wave pulses down a string which is fixed at one end. Waves will reflect off of the fixed end, travel back toward the source and interfere with the waves which have yet to hit the fixed end.}
	
	{Mathematically these frequencies are given by,}
		
		$$f_{n} = \frac{n}{2L}\sqrt{\frac{T}{\mu}}$$
		
	{Where, \textit{n} is the fundatmental frequency number, \textit{L} is the length of the string, \textit{T} is the tension in the string and $\mu$ is the linear density of the string. We know that,}
	
		$$\mu = \frac{M}{L} \approx 1.63\times 10^{-3}$$

	{Therefore, we have,}		
		
		$$f_{n} = \frac{n}{2L}\sqrt{\frac{mg}{\mu}} = f_{n} = \frac{n}{2.94}\sqrt{6.01\times 10^{3} m}$$

	{Where \textit{m}, is the mass of the object and \textit{n} is the fundamental frequency in question.}

			
			