

{}



\section{{Procedure I - In Equilibrium}}

{}

\begin{enumerate}
	\item {Connect the microphone to CH-1 of the LabPro computer interface.}
	\item {Open Logger Pro and check if your LabPro is properly connected, an icon should
appear in the top left of the screen if it is.}
	\item {Click Experiment → Zero to center the waveform on the axis.}
	\item {trike your tuning fork with a rubber mallet. Hold the microphone close to the fork. In the Logger Pro application, click ▶ . The computer will take data for just 0.05s to display the rapid pressure variations. The vertical axis is related to the variation in air pressure, but given in arbitrary units (you will not need to calibrate the scale in Pascals for this lab).}
	\item {Zoom in / scale your graph axes so that the period can be measured easily. Note that you can have your graphs automatically zoom by clicking the button. You can also zoom into a particular area by selecting an area on the graph and then clicking the button; the graph will zoom into the dark grey rectangle which you highlighted.}
	\item {sing your collected waveform and Logger Pro, determine the period of the wave. You can do this by selecting Analyze → Examine and dragging the mouse over the graph. You should be able to read the time interval you selected ($\Delta t$) in the lower left corner. Record the period, and an estimate of its uncertainty.}
	\item {Determine the amplitude of the wave (in arbitrary units) using the same method as above. Be careful in reading your graph if the sinusoidal curve is not centered on the x-axis. If this is the case, the easiest way to determine the amplitude is to determine the vertical distance between a peak and a trough and divide by two. Record the amplitude and a measure of its uncertainty.}
	\item {Save the data by choosing Experiment → Store Latest Run. Hide the run by
choosing Data → Hide Data Set, and selecting Run 1.}
\end{enumerate}

\section{{Procedure II - In Oscillation}}

{}

\begin{enumerate}
	\item {Connect the microphone to CH-1 of the LabPro computer interface.}
	\item {Open Logger Pro and check if your LabPro is properly connected, an icon should appear in the top left of the screen if it is.}
	\item {Click Experiment → Zero to center the waveform on the axis.}
	\item {Learn how to play a tone with a tone generator:}
	\begin{enumerate}
		\item {Make sure your computer’s volume is set quite low. You can always increase the volume if the tone is too soft, but playing an extremely loud tone is unpleasant for everyone!}
		\item {Open the program Audacity which you can use to generate pure tones. To generate a tone, select Generate → Tone. Leave the waveform as “Sine”, enter the frequency and the amplitude (a number between 0 and 1), and select the duration (we recommend at least 10 seconds). Then click Generate Tone }
		\item {In the audacity program click the green arrow ▶ to play the tone.}
	\end{enumerate}
	\item {Working with the group across from you, set up two tones to be played simultaneously using Audacity. The frequencies of the two tones should differ by no more than 10 Hz, and should have (roughly) equal volume. Play the two tones together.}
	\item {Again, working with the group across from you, set up two tones to be played simultaneously, but make the frequency of the two tones differ by 50-100 Hz.}
	\item {Play the two tones (which differ by 50-100 Hz) together, and place your microphone in a position where both tones can be heard. While the tones are playing, switch back to Logger Pro and collect data by pressing ▶ .}
	\item {As you did before, select Analyze → Examine. Determine the beat period (the time between successive beats)}
	\item {Store this run by choosing Experiment → Store Latest Run.}
\end{enumerate}


