
	{In this particular lab experiment, the component of physics known as circular motion was introduced. This area of physics analyzes the fundamental behavior and laws that an object adhere's to when the object moves around a circular path. An important part of this is circular acceleration, or as it’s known as centripetal acceleration. Therfore we have,}

		$$a_{c} = \frac{v^{2}}{r}$$

	{Where $v$ is the velocity and $r$ is the radius of the circular motion.}

	{Combining Newton's 2nd law with the above experession, we have,}

		$$F = ma = \frac{mv^2}{r}$$

	{The circular motion also comprises circular velocity, which is the velocity of the object moving
along the circumference of a circle. Through the circular velocity the relationship between
velocity and period(T) is found to be,}

		$$v = \frac{\Delta x}{\Delta t}$$

	{The equation can be expressed in terms of a circle. For example, the distance(\textit{x}) can be further denoted as the perimeter of the the circle’s circumference which $2\pi r$. The time(\textit{t}) can also be denoted as period(\textit{T}) which is the time taken to complete one cycle specified by a given point. Therfore we have,}

		$$v = \frac{\Delta x}{\Delta t} = \frac{2\pi r}{T}$$

	{In this experiment to determine the spring force($F_{s}$), the mass meter was used. The mass meter was attached to the mass(\textit{m}), and the spring and the mass were pulled directly over the pin; the readings of the mass determined by the mass meter was denoted as \textit{M}.}

	{In order to calculate the force of the spring, the \textit{M} value must be converted into $F_{s}$. To do this, it must be understood that the mass meter is no different than a hanging scale. A hanging scale measures the force of gravity of an object from the object’s mass, the mass is hung on the scale and it is measured. Similarly, in this situation, the mass was hung on the scale, but instead a pulled force was applied, however the same instrument mechanics were applied. This implies that,}

		$$F_{s} = Mg$$

	{As it could be seen from the experiment, the results were calculated with fixed variables, to ensure accuracy. For example, for each time variable determined, there were 20 revolutions made. So although time is given, it is based on time required for 20 revolutions. However, period(\textit{T}) is defined by the time required for a single revolution.}

	{Using the given data, for the three main variables, T(period), R(radius), and M(mass of object stretched to equilibrium with applied force), we can establish the relationship between them, and express it through a graph. The following relationship requires a equation that combines all of these elements into a single equation. Therefore we have,}

		$$Mg = \frac{mv^2}{r} = \frac{\frac{4m\pi^2 r^2}{T^2}}{r} = \frac{4m\pi^2 r}{T^2} $$

		$$\implies M = \frac{4m\pi^2 r}{gT^2}$$

	{Thus, the relation between the three variables is solved with \textit{M} on one side, and the other two
variables on the other side. However to demonstrate the relationship using a graph we must
separate the equation to the graph form of $y = mx + b$. Where the M is proportional to the
variables \textit{T} and \textit{r} by some factor, that is the slope \textit{m}.}

\begin{table}[H]
\centering
\begin{tabular}{cccc}
\hline
\multicolumn{1}{c}{$\frac{r}{T^2}$(m/s\textasciicircum{}2)} & Error propagation($\frac{r}{T^2}$) & \textit{M} value(Kg) & Error propagation(\textit{M}) \\ \hline
0.228                                                                       & 0.123                                     & 0.58        & 0.05                 \\
0.323                                                                       & 0.162                                     & 0.69        & 0.05                 \\
0.30                                                                        & 0.135                                     & 0.795       & 0.05                 \\
0.4090                                                                      & 0.221                                     & 0.86        & 0.05                 \\ \hline
\end{tabular}
\end{table}

	\includegraphics[width=\textwidth]{Lab	 4.png}

	{To be able to find the predicted slope of the graph, for the constant ‘m’ value, the previous
combined equation must be examined. Through the formula for the relationship between M, T
and R, in the slope-intercept form for a straight line, the slope value can be determined as,}

		$$M = \frac{4m\pi^2 r}{gT^2}$$

	{Therefore we have,}

		$$m = 1.8531 \frac{kgs^2}{m}$$

	{As it could be seen from the results, the measured slope based on our graph is 1.4361, while the predicted slope which is derived from various physics concepts and laws is 1.8531.}
	
	{Although the results have a small difference, there were multiple factors that accounted for this. Such as inaccurate readings, instrumental inaccuracies, and unintended interference in the lab. The percentage error is figured out to be,}

		$$\Delta M = 17\%$$

	{However, as evidently proved the slope determined from the lab experiment with the practical results well complemented the theoretical and mathematical approach to calculate the slope.}
	
	{Both methods relied on fundamental concepts, and abided by the physics laws, thus both methods are valid and play a key role in determining the results.}