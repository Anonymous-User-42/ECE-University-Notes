{Before beginning this investigation we must know the methods and the theory behind the investigation and the due course of action.}

\section{Mass Measurment}

	{We shall be measuring the mass of the unknown metal block by the use of a triple beam mechanical balance scale}

\section{Volume Measurement}

	{The volume of the metal block can be measured by various ways:}
	
	\begin{itemize}
    		\item {Vernier Caliper}
    		\item {Ruler}
    		\item {Archimedes Principle}
    		\label{vol}
	\end{itemize}	
	
	\subsection{Volume Measurement by Vernier Caliper}
		
		{The volume can be measured by induvidually measuring the dimensions of the meatal block and then simply multiyplying those dimensions to obtain the volume of the metal block.}		
		
			$$V = l \times b \times h$$		
		
		{Where \textit{V} is the volume, \textit{l} is the length, \textit{b} is the breadth and \textit{h} is the height of the metal block.}		
		
	\subsection{Volume Measurement by Ruler}
		
		{The measurement process of the volume of the metal block by ruler is the same as the one done using the vernier caliper.}
		
		{The difference these both instruments have is the accuracy in measurement of volume.}		
		
	\subsection{Volume Measurement by Archimedes Principle}
		
		{The volume can be measured by using the archimedes principle by immersing the metal block in a beaker filled with fluid (water) and calculating the difference of the fluid levels before and after immersion.}
		
\section{{Density}}
        
    {Once we find out the \textbf{mass} and \textbf{volume} of the metal block with its subsequent uncertainities, we can finally find the \textbf{density} (the aim of the experiment) of the metal block.}
        
    {\textbf{Density} is a function that is mathematically defined as,}
        
    $$\rho = \frac{m}{V}$$
        
    {Where, \textit{m} is the \textbf{mass} of the metal block and \textit{V} is the \textbf{volume} of the metal block.}
        
	{\textbf{Note}: After measuring the mass and the volume of the metal block, we can find the density of the metal block, by dividing both the quantities.}   


