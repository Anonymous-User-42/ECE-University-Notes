

{This laboratory experiment helped examine the various outcomes that may be attained while dealing with elastic and perfectly inelastic collisions. The carts remained grouped together and always had the same ultimate velocity during the perfectly inelastic collision. Greater $m_{1}$ values would lead to a faster ultimate velocity, while greater $m_{2}$ values would result in a considerably slower velocity between the two carts. The two carts were able to repel one another and not "stuck together" due to the elastic collision.}

{This led to The fact that cart one only experienced a "negative" velocity (bounced backwards) during the elastic impact is another intriguing feature of this experiment. This happened when cart two (the stationary cart) had a mass significantly higher than cart 1. This makes sense given that cart two, compared to the other two trail runs, is much harder to move forward and necessitates more kinetic energy. Also, the KE lost during inelastic collisions is significantly higher than the momentum lost.}

{This is because, theoretically, momentum is stored during inelastic collisions, and KE is lost. In elastic collisions, the momentum lost and the kinetic energy lost is lower than inelastic collisions. This is because, theoretically, during elastic collisions, momentum and KE are conserved. These different values occur due to slight human and lab errors that do not get the exact value as expected by theory.}

{}

{}

{}

{}

{}

{}

{}


