

{}

\section{{Experiment 1}}

{}

\begin{enumerate}
	\item {Connect the motion detector software to the computer’s hardware, ensure all electronics are switched on and in good working order}
	\item {Utilizing the lab jack and the protractor, angle the track at 15 degrees}
	\item {Adjust the motion detector and cart, so they will not be closer than 15cm from each other}
	\item {Take the cart to the initial position and prepare to release}
	\item {Press play, then release the block. If performed correctly, you should see two graphs on the logger pro software}
	\item {In the data seems inaccurate, try adjusting the cart and re-attempting}
	\item {Using the software to form a line of best fit, then record the data}
	\item {Clear the data once saved and repeat it three more times}
	\item {Save and Export the data once done}
\end{enumerate}

\section{{Experiment 2}}

{}

\begin{enumerate}
	\item {Ensure all technological equipment is switched on and functioning just as before \textbf{Experiment 1}}
	\item {Position the cart at the bottom of the ramp, then push it up the ramp hard enough that it reaches the top but not too hard that it hits the sensor}
	\item {Using logger pro press play and record the data}
	\item {Using the software create a linear best fit}
	\item {Save and export the data once done}
\end{enumerate}

\section{{Experiment 3}}

{}

\begin{enumerate}
	\item {Open the bouncing ball video provided}
	\item {Watch the video a few times till you get a general understanding of the motion of the ball}
	\item {Utilizing the pause, play, forward and rewind controls advance the video to an instance where the ball leaves contact with the ball for the first time}
	\item {Play the video and use a painting tool to plot the parabola-shaped motion of ball ascent and descent}
	\item {Once completed, fit a cure to the data on the x data and y data}
	\item {Save and export the data once done}
\end{enumerate}
