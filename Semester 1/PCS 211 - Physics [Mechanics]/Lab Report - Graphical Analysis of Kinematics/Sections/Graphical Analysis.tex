\section{{Graphical Analysis}}
            
    \subsection{{Graphs in relation to the Experimental values versus time}}
            
        \subimport{./}{ExpGraph}
            
    \subsection{{Graphs in relation to the Simulation values versus time}}
            
        \subimport{./}{SimGraph}
            
            
    \textit{Upon close visual observation, we see that the difference in the plotted values of that of the \textbf{simulation} and \textbf{experimental values} of the radial displacement, angular displacement, angular frequency and absolute frequency from the \textbf{x-t} graph are very \textbf{minute} to the extent that it would be right to say and consider that the experimental values are both \textbf{accurate} and \textbf{precise} in relation to that of the \textbf{literature/theoretical/simulation values}.}
        
	\textit{It is evident from studying the system \textbf{numerically} and \textbf{graphically} that the system exhibits an \textbf{exponential decay} with time for all parameters that are being investigated. This points out to \textbf{one conclusion}, that \textbf{all parameters of the system undergoes damping}, according to their various \textbf{inert energies}.}
	 
	\textit{This \textbf{contradicts} the \textbf{initial hypothesis} laid out prior to beginning the investigation as if, the system had to model \textbf{similarly} to the simple pendulum, then the damping effect would take place on only the \textbf{angular motion} component of the system, but we see that this is not the case.}        
        
    \textit{We also observe that, with increase in mass, the \textbf{amplitude} of the parameters employed in this investigation also increases, while the \textbf{frequency} of the parameters employed stays approximately constant.}
        
    
    

