{Before beginning this investigation we must know the methods and the theory behind the investigation and the due course of action.}

\section{{Newton's 2nd Law}}

	{Newton's 2nd Law of Motion states that When a body is acted upon by a force, the time rate of change of its momentum equals the force. Mathematically,}

		$$F = \frac{\Delta(mv)}{\Delta t} = m\frac{\Delta v}{\Delta t} = ma$$

\section{{Mass}}

	{The mass has an inverse relationship with the acceleration of the cart, which is proportional to the force exerted on the cart. Newton's second law also requires the entire mass of the system to be accounted into for the law to hold true. The mass is the sum of all masses. Mathematically,}

		$$M = m_{1} + m_{2}$$

	{Where, $m_{1}$ is the mass of the hanger and $m_{2}$ is the mass of the glider.}

\section{{Tension \& Forces}}

	{If we assume $F_{x}$ is the horizontal force and $F_{y}$ is the vertical force. The total force of our system is equal to the sum of the horizontal force acting on the cart and the net vertical force acting on the hanging mass. Tension ($T$) is the sole horizontal force acting on the rope. According to Newton's second law, the tension is equal to the glider's mass multiplied by its acceleration, mathematically expressed as:}
	
		$$F_{x} = T = m_{2}a$$	
	
	{The normal force, $F_{n}$, which is the same as the tension, is subtracted from the gravitational force, Fg (Tension is opposite in direction), which is calculated as hanging mass multiplied by the gravitational force ($T$). Mathematically,}	
	
		$$F_{y} = F_{g} - F_{n} = m_{1}g - T = m_{1}a$$	
		
\section{{Acceleration}}

	{With algebraic manipulation of Newton's law we have,}
	
		$$a = \frac{F}{m}$$	
	
	{Therfore we have,}

		$$a = \frac{m_{1}g}{m_{1} + m{2}}$$

