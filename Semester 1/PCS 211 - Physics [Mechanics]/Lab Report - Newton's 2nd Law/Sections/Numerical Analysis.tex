\section{{Numerical Analysis}}
        
    \subsection{{Radial Displacement versus Time}}
            
                \textit{We initially defined and conditionalized the initial value of the radial displacement versus time to be equal to 0.}
                
                \textit{Due to viscous underdamped motion, it gains energy and momentum and preserves it. With gradual but steady loss in its energy state and is in motion for more than 300 seconds before coming to a complete halt. ie. no motion.}
    
    \subsection{{Angular Displacement versus Time}}
            
                \textit{We initially defined and conditionalized the initial value of the angular displacement versus time to be equal to $\pi/2$.}
                
                \textit{Due to viscous underdamped motion, it gains energy and momentum and preserves it. With gradual but steady loss in its energy state and is in motion for more than 120 seconds before coming to a complete halt. ie. no motion.}
    
                
    \subsection{{Angular Frequency versus Time}}
                
                \textit{We previously defined angular frequency ($\omega_\theta$) to be as equation \ref{eq6}.}
                
                \textit{The initial state ($\omega_\theta(0)$) that follows is,}
                
                $$\omega_\theta(0) = \sqrt{\frac{g}{l_0 + x(0)}} = \sqrt{\frac{g}{l_0}}$$
                
                \textit{But we had previously mentioned that the magnitude of rest length of the spring would be 1 meter ($l_0 = 1 m$). Therefore we have,}
                
                $$\omega_\theta(0) = \sqrt{g} \approx 3.13209195$$
                
                \textit{This parameter keeps fluctuating through the time domain as this is a function of radial displacement which in turn is a function of time.}
                
                \textit{This initial state defined and derived above is uniform irrespective of the mass employed}
                
    
    \subsection{{Absolute Frequency versus Time}}
                
                \textit{We previously defined absolute frequency ($\omega_r$) to be as equation \ref{eq5}.}
                
                \textit{The initial state ($\omega_r(0)$) that follows is,}
                
                $$\omega_r(0) = \sqrt{\frac{k}{m}}$$
                
                \textit{This parameter does not fluctuate through the time domain as this is not a function of any arbitrary variable that would be in turn is a function of time.}
                
                \textit{This initial state defined and derived above is constant for any particular case with a steady constant mass m, throught a case}
            
            


