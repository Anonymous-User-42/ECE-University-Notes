

{Without a doubt these crucial physical quantities are an important aspect of physics, and these variables set the principles for circular motion. The relationship between the three main quantitative variables, T(period), M(mass of object stretched to equilibrium with applied force), and R(radius) is the core concept that outlines circular motion. In order to successfully build, test and create a structure, the principles of circular motion must be understood thoroughly.}

{As a matter of fact, to design the centrifuge that helps astronauts prepare in proper training, the circular motions must be placed into great consideration. The goal of an effective centrifuge is to have the right rotation period, one that is not excessive to the spring, but also with high-speed capabilities for training. From the results, it could be very well concluded that as rotational periods get smaller, the less time it takes to do a single revolution, which implies that the object gets quicker and both velocity and acceleration increase dramatically. This also means that more force is present, and force is needed to get the rotation to spin appropriately.}

{As a result, based on this experimental situation, the rotation period that will give the astronauts the appropriate adequate training and will not overstress the spring is a period no less than 0.717 (seconds/Revolutions). Any rotation period below that can be considered excessive, as it not only requires a lot more force that is hard to achieve, but it also requires very high speeds to maintain a constant circular movement over the designated pin. Using the circular motion principles of physics, the very behavior of an object moving around a circumference can be understood and
can be well applied to the real world.}

{}

{}

{}

{}

{}

{}

{}


