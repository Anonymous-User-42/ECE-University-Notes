{The primary objective of this design project is to develop and construct a Bipolar Junction Transistor (BJT) amplifier that adheres to a set of specific performance criteria. The circuit was required to meet the following key specifications: a power supply of \( +10 \, \text{V} \) relative to the ground, with a quiescent current draw not exceeding \( 10 \, \text{mA} \). The amplifier's no-load voltage gain at 1 kHz was targeted at \( |A_{vo}| = 50 \) with a tolerance of \( \pm 10\% \), and the maximum no-load output voltage swing at 1 kHz was required to be at least \( 8 \, \text{V}_{\text{p-p}} \). Additionally, when loaded with a \( 1 \, \text{k}\Omega \) resistor, the loaded voltage gain at 1 kHz was expected to be no less than \( 90\% \) of the no-load voltage gain, and the maximum loaded output voltage swing was required to be no smaller than \( 4 \, \text{V}_{\text{p-p}} \).}

{The input resistance at 1 kHz was specified to be no less than \( 20 \, \text{k}\Omega \). The design allowed for either inverting or non-inverting amplifier configurations, with a frequency response spanning from \( 20 \, \text{Hz} \) to \( 50 \, \text{kHz} \) (−3 dB response). The amplifier was to be constructed using BJT transistors, with a limitation of no more than three stages. Additionally, the design was restricted to using resistances with values smaller than \( 220 \, \text{k}\Omega \) from the E24 series, and capacitors with values of \( 0.1 \, \mu\text{F} \), \( 1.0 \, \mu\text{F} \), \( 2.2 \, \mu\text{F} \), \( 4.7 \, \mu\text{F} \), \( 10 \, \mu\text{F} \), \( 47 \, \mu\text{F} \), \( 100 \, \mu\text{F} \), or \( 220 \, \mu\text{F} \). Other components, such as additional BJTs, diodes, and Zener diodes, were to be sourced exclusively from the ELE404 lab kit.}

{}

{}

{}

