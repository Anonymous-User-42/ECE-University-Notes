\begin{lstlisting}[language=VHDL, caption={VHDL code for the General Purpose ALU}]
library IEEE;
use IEEE.std_logic_1164.all;
use IEEE.std_logic_unsigned.all;
use IEEE.numeric_std.all;

entity ASU is
    port(
        Clock: in std_logic;
        A, B: in unsigned(7 downto 0);
        studentId: in unsigned(3 downto 0);
        oP: in unsigned(15 downto 0);
        neg: out std_logic;
        R1: out unsigned(3 downto 0);
        R2: out unsigned(3 downto 0)
    );
end ASU;

architecture calculation of ASU is
    signal reg1, reg2, result: unsigned(7 downto 0) := (others => '0');
    signal reg3: unsigned(0 to 7);
    begin
        reg1 <= A;
        reg2 <= B;
        process(Clock, oP)
        begin
            if(rising_edge(Clock)) then 
            case oP is
                when "0000000000000001" =>
                    result <= reg1 + reg2;
                when "0000000000000010" =>
                    if (reg2 > reg1) then
                            result <= (reg1 + (not reg2));
                        else
                            result <= reg1 - reg2;
                            neg <= '0';
                        end if;
                when "0000000000000100" =>
                    result <= not reg1;
                when "0000000000001000" =>
                    result <= reg1 nand reg2;
                when "0000000000010000" =>
                    result <= reg1 nor reg2;
                when "0000000000100000" =>
                    result <= reg1 and reg2;
                when "0000000001000000" =>
                    result <= reg1 xnor reg2;
                when "0000000010000000" =>
                    result <= reg1 or reg2;
                when "0000000100000000" =>
                    result <= reg1 xnor reg2;
                when others =>
                
            end case;
        end if;
    end process;
    R1 <= result(3 downto 0);
    R2 <= result(7 downto 4);
end calculation;
\end{lstlisting}